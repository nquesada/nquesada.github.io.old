\documentclass[12pt]{article}

%                   G. 't Hooft  macros version 2000

\newread\testifexists
\def\GetIfExists #1 {\immediate\openin\testifexists=#1
    \ifeof\testifexists\immediate\closein\testifexists\else
    \immediate\closein\testifexists\input #1\fi}

%\usepackage{gthstyle}
\mathsurround=1pt \parskip=5pt

\def\epsffile#1{Figure: #1}     % (in case epsf does not exist)
\GetIfExists  epsf
    %  use:     \epsfxsize=<dimen> \epsfbox{filename.ps}
    %  or:      \epsfysize=<dimen> \epsfbox{filename.ps}

\def\Bbb#1{\setbox0=\hbox{$\tt #1$}  \copy0\kern-\wd0\kern .1em\copy0}
    % (in case amssym does not exist)

\def\bbf#1{\setbox0=\hbox{$#1$} \kern-.025em\copy0\kern-\wd0
        \kern.05em\copy0\kern-\wd0 \kern-.025em\raise.0433em\box0}
    % boldface in math mode.

\GetIfExists  amssym.def          %% for \Bbb A - Z %%

\def\a{\alpha}      \def\b{\beta}   \def\g{\gamma}      \def\G{\Gamma}
\def\d{\delta}      \def\D{\Delta}  \def\e{\varepsilon}
\def\h{\eta}         \def\k{\kappa} \def\l{\lambda}     \def\L{\Lambda}
\def\m{\mu}         \def\f{\phi}    \def\F{\Phi}        \def\vv{\varphi}
\def\n{\nu}         \def\j{\psi}    \def\J{\Psi}
\def\r{\varrho}     \def\s{\sigma}  \def\SS{\Sigma}
\def\t{\tau}        \def\th{\theta} \def\thh{\vartheta} \def\Th{\Theta}
\def\x{\xi}         \def\X{\Xi}     \def\ch{\chi}
\def\w{\omega}      \def\W{\Omega}  \def\z{\zeta}

\def\HH{{\cal H}} \def\LL{{\cal L}} \def\OO{{\cal O}}
\def\pa{\partial} \def\ra{\rightarrow}
\def\na{\nabla}
\def\bal{$\bullet$} \def\bel{$\circ$}
\def\dd{{\rm d}}  \def\bra{\langle}   \def\ket{\rangle}

\def\qu{\ {\buildrel {\displaystyle ?} \over =}\ }
\def\deff{\ {\buildrel{\rm def}\over{=}}\ }

\newcommand{\ft}[2]{{\textstyle\frac{#1}{#2}}}
\def\fract#1#2{{\textstyle{#1\over#2}}}
\def\ffract#1#2{\raise .3 em\hbox{$\scriptstyle#1$}\kern-.25em/
                \kern-.2em\lower .2 em \hbox{$\scriptstyle#2$}}
\def\fractje#1#2{{\scriptstyle{#1\over#2}}}
\def\half{\fract12} \def\quart{\fract14} \def\halff{\ffract12}
\def\halfje{\fractje12}
\def\part#1#2{{\partial#1\over\partial#2}}
\def\ex#1{e^{\textstyle#1}}

\newcommand{\dddot}[1]{{\buildrel{\displaystyle{\,...}}\over #1}}

\def\iz{\quad = \quad}
\newcommand{\dys}{\displaystyle}
\newcommand{\scs}{\scryptstyle}
\newcommand{\txs}{\textstyle}


\newcommand{\tl}[1]{\tilde{#1}}
\renewcommand{\^}[1]{\hat{#1}}
\newcommand{\sm}[1]{\mbox{\scriptsize #1}}
\newcommand{\tn}[1]{\mbox{\tiny #1}}
\newcommand{\Tr}{{\mbox{Tr}}\,}
\newcommand{\be}{\begin{eqnarray}}
\renewcommand{\le}[1]{\label{#1}\end{eqnarray}}
\newcommand{\ee}{\end{eqnarray}}
\newcommand{\eqn}[1]{(\ref{#1})}
\newcommand{\nn}{\nonumber\\}
\newcommand{\nm}{\nonumber}
\newcommand{\rf}[1]{\cite{ref:#1}}
\newcommand{\rr}[1]{\bibitem{ref:#1}}

\newcommand{\nl}{\newline}
\newcommand{\cl}{\centerline}
\newcommand{\fn}{\footnote}
% \def\tcl{\textcolor}    \def\clb{\colorbox}
\setlength{\oddsidemargin}{0in} \setlength{\textwidth}{6.3in}
\setlength{\topmargin}{-0.5in} \setlength{\textheight}{9in}

\renewcommand{\thesection}{\arabic{section}}
\renewcommand{\theequation}{\thesection.\arabic{equation}}
\newcommand{\newsec}[1]{\section{#1}\setcounter{equation}{0}}
    % Counts equations as (section#, eq#)

\begin{document}

\begin{titlepage}

\title{ \Large\bf
SPECIAL FUNCTIONS and POLYNOMIALS\\}

\author{\small Gerard 't~Hooft\\
\small Stefan Nobbenhuis}
\date{\small Institute for Theoretical Physics \\
Utrecht University, Leuvenlaan 4\\ 3584 CC Utrecht, the
Netherlands\medskip \\ and
\medskip \\ Spinoza Institute \\ Postbox 80.195 \\ 3508 TD
Utrecht, the Netherlands \smallskip}

\maketitle

\begin{quotation} \noindent  Many of the special functions
and polynomials are constructed along standard procedures In this
short survey we list the most essential ones.
\end{quotation}

\vfill \flushleft{\today}

\end{titlepage}

\eject

%%%%%%%%%%%%%%%%%%%%%%%%%%%%%%%%%%%%%%%%%%%%%%%%%%%%
\newsec{Legendre Polynomials \(P_\ell(x)\)\ .}
Differential Equation: \be (1-x^2)\,P_\ell\, ''(x)-2x\,P_\ell\,
'(x)+\ell (\ell+1)\,P_\ell(x)\iz 0\ ,\nm\ee or \be {\dd\over\dd
x}(1-x^2){\dd\over\dd x}\,P_\ell(x)+\ell(\ell+1)\,P_\ell(x)\iz 0\
.\ee

\bigskip\noindent Generating function:\be
\sum_{\ell=0}^\infty\,P_\ell(x)t^\ell&=&(1-2xt+t^2)^{-\halfje}\
\quad\mbox{for}\quad |t|<1, |x|\leq 1. \ee
 Orthonormality: \be\int_{\raisebox{-.3em}{-1}}^{\raisebox
{.3em}{1}} P_\ell(x)\,P_{\ell\,'}(x)\,\dd x&=&{2\over
2\ell+1}\,\d_{\,\ell\,\ell\,'}\ ,\\ \sum_{\ell=0}^\infty
P_\ell(x)P_\ell(x')(2\ell+1)&=&2\d(x-x')\ .\ee
 Expressions for\(P_\ell(x)\)\,: \be P_\ell(x)&=&{1\over
2^\ell}\sum_{\n=0}^{[\ell/2]}\,{(-1)^\n \, (2\ell-2\n)!\over
\n!\,(\ell-\n)!\,(\ell-2\n)!}\,x^{\ell -2\n}\\ &=&
{1\over\ell!\,2^\ell}\Big({\dd\over\dd x}\Big)^\ell\,(x^2-1)^\ell\
,\\&=&{1\over\pi}\int_0^\pi(x+\sqrt{x^2-1}\cos\vv)^\ell\,\dd\vv\,
. \ee
 Recurrence relations: \be \ell\,P_{\ell-1}-(2\ell+1)\,x\,
P_\ell+(\ell+1)\,P_{\ell+1} = 0\ ; \nn P_\ell =
x\,P_{\ell-1}+{x^2-1\over\ell}\,P\,'_{\ell-1}\ \ ; \nn xP\,'_\ell\
-\ \ell\,P_\ell = P\,'_{\ell-1}\ \ ;\nn
xP\,'_\ell+(\ell+1)\,P_\ell&=& P\,'_{\ell+1}\ \ ;\nn {\dd\over\dd
x}\left[P_{\ell+1}-P_{\ell-1}\right] = (2\ell+1) \, P_\ell .\ee
 Examples: \be P_0=1\ ,\qquad P_1=x\ ,\qquad
P_2=\half(3x^2-1)\ ,\qquad P_3=\half x(5x^2-3)\ .\ee \pagebreak[4]

\newsec{Associated Legendre Functions \(P_\ell^m(x)\)\ .}
Differential equation: \be
(1-x^2)\,P_\ell^m(x)\,''-2x\,P_\ell^m(x)\,'+\Big(\ell(\ell+1)-{m^2\over
1-x^2}\Big)\,P_\ell^m(x)&=&0\ .\ee
 Generating function: \be\sum_{\ell=0}^\infty\,\sum_{m=0}^\ell\,
{P_\ell^m(x)\,z^m\,y^\ell\over m!}&=&\left[1-2y\left(x+z
\sqrt{1-x^2}\, \right)+y^2\right]^{-\half}\ .\ee
 Orthogonality: \be\int_{\raisebox{-.3em}{-1}}^{\raisebox {.3em}{1}}
P_\ell^m(x)\,P_{\ell\,'}^m(x)\,\dd x={2\over
2\ell+1}\,{(\ell+m)!\over (\ell-m)!}\,\d_{\,\ell\,\ell\,'}\
,\qquad (\,\ell,\,\ell\,'\geq m\,)\ .\ee \be \sum_{\ell=m}^\infty
(2\ell+1){(\ell-m)!\over(\ell+m)!}\,P_\ell^m(x)\,P_\ell^m(x')=
2\d(x-x')\ ,\qquad(\,|x|<1\hbox{ and }|x'|<1\,)\ .\ee
 \noindent Expressions for \(P^m_\ell(x)\)\footnote{Note that some authors define $P_\ell^m(x)$ with a factor
 $(-1)^m$, giving $P_\ell^m(x)=(-1)^m(1-x^2)^{\halfje m}\left(\dd\over\dd x\right)^m\,P_\ell(x)$. Obviously this minus
sign propagates to the generating function, the recurrence
relations and the explicit examples, when $m$ is odd.}: \be
P_\ell^m(x)&=&(1-x^2)^{\halfje m}\left(\dd\over\dd
x\right)^m\,P_\ell(x)\ .\ee \be P_\ell^m(x)&=&
{(\ell+m)!\over\ell!\,\pi}(-1)^{m/2}\int_0^\pi\Big(x+\sqrt{x^2-1}\cos\vv\,\Big)^\ell\cos
m\vv\,\dd\vv\ .\ee
 Recurrence relations: \be P_{\ell}^{m+1} -
 \frac{2mx}{\sqrt{1-x^2}}P_{\ell}^m +
 \{\ell(\ell+1)-m(m-1)\}P_{\ell}^{m-1} = 0\ee
\be \sqrt{1-x^2}\,P_\ell^{m+1}(x)&=&(1-x^2)\,P^m_\ell(x)\,'+ mx
\,P_\ell^m(x)\ ,\nm\ee \be (2\ell+1)x\,P_\ell^m
=(\ell+m)\,P_{\ell-1}^m+(\ell+1-m)\,P_{\ell+1}^m\ ,\ee \be
x\,P_\ell^m &=& P_{\ell-1}^m-(\ell+1-m)\sqrt{1-x^2}\,P_\ell^{m-1}\
,\nm\ee \be
P^m_{\ell+1}-P^m_{\ell-1}&=&(2\ell+1)\,P_\ell^{m-1}\sqrt{1-x^2}\
,\ee and various others. \par \noindent
 Examples: \be P_1^1=\sqrt{1-x^2}
&,\qquad& P_2^2=3(1-x^2)\ ,\nn P^1_2=3x\sqrt{1-x^2}&,\qquad&
P^2_3=15\,x(1-x^2)\ .\ee  \pagebreak[4]

\newsec{Bessel \(J_n(x)\) and Hankel \(H_n(x)\) functions.}
Differential equation (for both \(J_n\) and \(H_n\)): \be
x^2\,J_n''(x)+x\,J_n'(x)+(x^2-n^2)\,J_n(x)&=&0\ .\ee
 Generating function (if \(n\) integer): \be \sum_{n=-\infty}^\infty
J_n(\alpha x)\,\left(\frac{s}{\alpha}\right)^n&=&\ex{\fract x2\big(s-\frac {\alpha^2}s\big)}\ ,\\
J_{-n}=(-1)^n\,J_n\ .\nm\ee Orthogonality:
\be\int_0^\infty\,\xi\,J_n(\a\xi)\,J_n(\b\xi)\,\dd\xi ={ 1\over
\a}\,\d(|\a|-|\b|)\ .\ee
\be\int_0^a\,\xi\,J_n(\a\xi)\,J_n(\b\xi)\,\dd\xi ={ a^2\over
2}\,\{J_{n+1}(\a a)\}^{2}\d_{\a\b}\ .\ee if in the $2^{nd}$
relation $\a,\b$ are roots of the equation $J_n(\a\xi)=0$.


 Expressions for \(J_n(x)\) (for \(n\)
integer): \be J_n(x)&=&\sum_{k=0}^\infty{(-1)^k\over
k!\,(k+n)!}\,\Big(\,{x\over2}\,\Big)^{n+2k}\iz {1\over 2\pi
i}\,\Big(\,{x\over2}\,\Big)^n\oint t^{-n-1}\dd t\,e^{\,t- x^2/4t}
\ . \\ J_n(x)&=& {1\over\pi}\int_0^\pi\cos\big(n\th-x\,
\sin\th\big)\,\dd\th\ . \ee
 Recurrence relations (for both \(J_n\) and \(H_n\)):
 \be\frac{d}{dx}\{x^n\,J_n(x)\}&=& x^n\,J_{n-1}(x)\ ;\nn
  J_{n-1}(x)+J_{n+1}(x) &=& {2n\over x}\,J_n(x)\ ;\nn
 J\,'_n(x)\iz J_{n-1}(x)-{n\over x}\,J_n(x) &=& {n\over x}\,J_n(x)
-J_{n+1}(x) = \vphantom{n\over x}
\half\big(J_{n-1}(x)-J_{n+1}(x)\big)\
 .\ee
  Relations between Hankel and Bessel functions: \be
H^{(1)}_n(x)&=&{i\over \sin n\pi}\,\Big(e^{-n\pi
i}J_n(x)-J_{-n}(x)\Big)\ ;\nn H^{(2)}_n(x)&=&{-i\over \sin
n\pi}\,\Big(e^{n\pi i}J_n(x)-J_{-n}(x)\Big)\ ,\ee so that \be
J_n(x)&=&\half\Big(H^{(1)}_n(x)+H^{(2)}_n(x)\Big)\ ;\nn
J_{-n}(x)&=&\half\Big(e^{n\pi i}H_n^{(1)}(x)+e^{-n\pi
i}H_n^{(2)}(x)\Big)\ .\ee\pagebreak[4]

\newsec{Spherical Bessel Functions \(j_\ell(x)\).} Differential
equation: \be (xj_\ell)''+\Big(x-{\ell(\ell+1)\over
x}\Big)\,j_\ell=0\ .\ee Generating Function:
\be\sum_{\ell=0}^\infty{j_\ell(x)\,t^\ell\over
\ell!}=j_0\Big(\sqrt{x^2-2xt}\,\Big)\ .\ee Orthogonality: \be
\int_0^\infty x^2j_\ell(\a x)\ j_\ell(\b x)\ \dd x={\pi\over
2\a}\,\d(\a-\b)\ .\ee\be \int_{-\infty}^\infty j_\ell(x)\,
j_{\ell'}(x)\ \dd x={\pi\over 2\ell +1}\,\d_{\ell\ell'}\ .\ee
Expressions for \(j_\ell\)\,: \be j_\ell(x)=\sqrt{\pi\over 2x}\
J_{\ell+\halfje}(x) = (-1)^\ell
x^\ell\left(\frac{d}{xdx}\right)^\ell\frac{\sin x}{x}\ ,\ee \be
j_\ell(x) &=& {x^\ell\over 2^{\ell+1}\,\ell !}\,\int_{-1}^1
e^{ixs}(1-s^2)^\ell\,\dd s \nn &=& {2^\ell\ell!\over(2\ell+1)!}\,
x^\ell\,\bigg(1-{1\over 1!\,(\ell+\fract 32)}\,\Big({x\over2}\Big)^2+
{1\over 2!(\ell+\fract32)(\ell+\fract52)}\,\Big({x\over
2}\Big)^4-\dots\bigg)\ .\ee Recurrence relations: \be j_{\ell+1} =
\frac{\ell}{x}j_{\ell} -
j'_{\ell}=\frac{2\ell+1}{x}j_{\ell}-j_{\ell-1}.\ee Examples: \be
j_0(x)&=&{\sin x\over x}\ ;\nn j_1(x)&=&{\sin x\over x^2}-{\cos
x\over x}\ ;\nn j_2(x)&=&{3\sin x\over x^3}-{3\cos x\over x^2} -{\sin
x\over x}.\ee \pagebreak[4] \newsec{Hermite Polynomials \(H_n(x)\).}
Differential equation: \be H_n''(x)-2x\,H_n'(x)+2n\,H_n(x)=0\ ,\nm\ee
or \be {\dd^2\over\dd x^2}\Big(H_n(x)\,e^{-\halfje
x^2}\Big)+(2n-x^2+1)\,H_n(x)\,e^{-\halfje x^2}=0\ .\ee Generating
function: \be\sum_{n=0}^\infty H_n(x)\,s^n/n!=e^{-s^2+2sx}. \ee
Orthogonality: \be\int_{-\infty}^\infty H_n(x)\,H_m(x)\,
e^{-x^2}\dd x=2^nn!\sqrt\pi\,\d_{nm}\\
\sum_{n=0}^\infty H_n(x)\,H_n(y)/(2^n
n!)=\sqrt\pi\,\d(x-y)\,e^{x^2}\ .\ee More general:
\be\sum_{n=0}^\infty
H_n(x)\,H_n(y)\,s^n/(2^nn!)={1\over\sqrt{1-s^2}}\exp\Big({-s^2(x^2+y^2)+2sxy\over
1-s^2}\Big)\ .\ee Expressions for \(H_n\)\,: \be H_n(-x)
&=&(-1)^n\,H_n(x);\\
H_n(x)&=&(-1)^n\,e^{x^2}\Big({\dd\over\dd x}\Big)^n e^{-x^2}\ ;\\
H_n(x)&=&(-1)^{n/2}n!\sum_{k=0}^{n/2}(-1)^k\,{(2x)^{2k}\over(2k)!\,(\halfje
n-k)!}\ ,\qquad\quad\hbox{ if \(n\) even,}\nn
H_n(x)&=&\big(-1\big)^{n-1\over 2}n!\sum_{k=0}^{n-1\over
2}(-1)^k\, {(2x)^{2k+1}\over (2k+1)!\,\Big({n-1\over 2}-k\Big)!} \
,\quad\hbox{ if \(n\) odd.}\ee Recurrence relations: \be{\dd^m
H_n(x)\over\dd x^m}&=&{2^{m}n!\over(n-m)!}\,H_{n-1}(x)\ ,\\
x\,H_n(x)&=& \half H_{n+1}(x)+n\,H_{n-1}(x)\ ,\\
H_n(x)&=&\bigg(2x-{\dd\over\dd x}\bigg)\,H_{n-1}(x)\ .\ee
Examples: \be H_0(x)=1\ ,\qquad H_1(x)=2x\ ,\qquad H_2(x)=4x^2-2\
.\ee \pagebreak[4]

\newsec{Laguerre Polynomials \(L_n(x)\).}
Differential equation: \be x\,L_n''(x)+(1-x)\,L_n'(x)+n\,L_n(x)=0\
 . \ee Generating function:\footnote{It's important to note that sometimes different definitions are
 used for the Laguerre and Associated Laguerre polynomials, where
 the Generating Function has the form: $\sum_{n=0}^\infty L_n(x)\,z^n/n!={1\over 1-z}\,\ex{{-xz\over
 1-z}}$. In this case the Expressions given for $L_n$ should be multiplied by $n!$.} \be \sum_{n=0}^\infty L_n(x)\,z^n={1\over 1-z}\,\ex{{-xz\over 1-z}}\ .\ee Orthogonality:
\be\int_0^\infty L_n(x)\,L_m(x)\,e^{-x}\,\dd x=\d_{nm}\ .\ee
Expressions for \(L_n\)\,: \be L_n(x)&=&{e^x\over n!}
\Big({\dd\over\dd x}\Big)^n\,(x^n\,e^{-x})\nn &=&{(-1)^n\over n!}
\,\Big(x^n-{n^2\over 1!}\,x^{n-1}+{n^2(n-1)^2\over
2!}\,x^{n-2}-\dots\,+(-1)^nn!\Big)\
 .\ee Recurrence relation:
\be(1+2n-x)\,L_n-n\,L_{n-1}-(n+1)L_{n+1}=0\ ;\nn
x\,L'_n(x)=n\,L_n(x)-n\,L_{n-1}(x)\ .\ee Examples: \be L_0(x)&=&1\
;\nn L_1(x)&=&1-x\ ;\nn
L_2(x)&=&\frac{1}{2!}(x^2-4x+2).\ee\pagebreak[4]

\newsec{Associated Laguerre Polynomials \(L_n^k(x)\)\,.}
Differential equation:\be
x\,{L_n^k}''+(k+1-x)\,{L_n^k}'+n\,L_n^k=0\ .\ee Generating
function:\be \sum_{n=0}^\infty L_n^k(x)\,z^n={1\over
(1-z)^{k+1}}\,\ex{{-xz\over 1-z}}\
.\ee\be\sum_{k=0}^\infty\,\sum_{n=k}^\infty{L_n^k(x)\,z^nu^k\over
k!}={1\over 1-z}\,\exp\Big({-xz+u\over 1-z}\Big)\ .\ee\
Orthogonality: \be\int_0^\infty
L_n^k(x)\,L_m^k(x)\,x^k\,e^{-x}\,\dd x=\frac{(n+k)!}{n!}\d_{nm}\
.\ee Expressions for \(L_n^k\)\,:\be
L_n^k(x)=(-1)^k\Big({\dd\over\dd x}\Big)^k\,L_{n+k}(x)\ .\ee\be
L_n^k(x)={e^xx^{-k}\over n!} \Big({\dd\over\dd
x}\Big)^n\,(x^{n+k}\,e^{-x})\ .\ee
 Recurrence
relation: \be L_{n-1}^k\,(x)+L_n^{k-1}\,(x) = L_n^k\,(x)\ ;\nn
x\,L_n^{k'}\,(x) = n\,L_n^k\,(x)-(n+k)L_{n-1}^k\,(x)\ .\ee
Examples: \be L_0^k(x)&=&1\ ;\nn L_1^k(x)&=&-x+k+1\ ;\nn
L_2^k(x)&=&{1\over 2} \left[x^2-2(k+2)x+(k+1)(k+2)\right]\ ;\nn
L_3^k(x)&=&{1\over
6}\left[-x^3+3(k+3)x^2-3(k+2)(k+3)x+(k+1)(k+2)(k+3)\right]
.\ee\pagebreak[4]

\newsec{Tschebyscheff\protect\footnote{Transliterations Chebyshev and
Tchebicheff also occur.} Polynomials \(T_n(x)\).} Differential
equation: \be (1-x^2){\dd^2\over\dd x^2}\,T_n(x)-x{\dd\over\dd
x}\,T_n(x)+n^2\,T_n(x)=0\ .\ee
 Generating function: \be \sum_{n=0}^\infty
T_n(x)\,y^n={1-xy\over 1-2xy+y^2}. \ee Symmetry relation: \be
T_n(x)=T_{-n}(x).\ee
Orthogonality:\be\int_{-1}^1{T_m(x)T_n(x)\over\sqrt{1-x^2}}\dd
x=\left\{\begin{array}{ll}
\halfje\pi\d_{nm}&m,n\neq 0\\
\pi &m=n=0\end{array}\right.\ee Expression for \(T_n\): \be T_n(x)
=\cos(n\cos^{-1}x)\ee \be
T_n(x)=\halfje\left[\left\{x+i\sqrt{1-x^2}\right\}^n+\left\{x-i\sqrt{1-x^2}\right\}^n\right].\ee
Recurrence relation: \be T_{n+1}-2x\,T_n(x)+T_{n-1}=0\
\ee\be(1-x^2)T'_n(x)=-nx\,T_n(x)+n\,T_{n-1}(x).\ee Examples: \be
T_0(x)&=&1\ ;\nn T_1(x)&=&x\ ;\nn T_2(x)&=&2x^2-1\ ;\nn
T_3(x)&=&4x^3-3x .\ee \pagebreak[4]

\newsec{Remark.}All of the functions discussed here are special
cases of ``hypergeometric functions"\\ \({}_mF_n(a_1,a_2,\dots
a_m;b_1,b_2,\dots b_n;z)\) defined by: \be {}_mF_n(a_1,a_2,\dots
a_m;b_1,b_2,\dots b_n;z) =
\sum_{r=p}^{\infty}\frac{(a_1)_r(a_2)_r\dots(a_m)_rz^r}{(b_1)_r(b_2)_r\dots(b_n)_rr!},\ee
where \be (a)_r
\equiv\frac{\Gamma(a+r)}{\Gamma(a)};\quad\quad\mbox{$r$ a positive
integer.} \ee Differential equations:\\ $m=n=1$: \be
z\Big({\dd\over\dd z}\Big)^2{}_1F_1+(b-z){\dd\over\dd
z}{}_1F_1-a\,\,{}_1F_1=0\ .\ee $m=2,n=1$:\be
z(1-z)\Big({\dd\over\dd
z}\Big)^2{}_2F_1+\Big(c-(a+b+1)z\Big){\dd\over\dd
z}{}_2F_1-ab\,\,{}_2F_1=0\ .\ee We have: \be P_\ell(x)
&=& {}_2F_1\left(-\ell,\ell+1;1;{1-x\over2}\right)\ ;\nn\\
P_\ell^m(x) &=&
\frac{(\ell+m)!}{(\ell-m)!}\frac{(1-x^2)^{m/2}}{2^mm!}{}_2F_1\left(m-\ell,m+\ell+1;m+1;{1-x\over2}\right)\
;\nn\\
J_n(x)&=&\frac{e^{-ix}}{n!}\left(\frac{x}{2}\right)^n{}_1F_1(n+\halfje;2n+1;2ix)\
;\nn\\
H_{2n}(x)&=&(-1)^{n}\frac{(2n)!}{n!}{}_1F_1(-n;\halfje;x^2)\
;\nn\\
H_{2n+1}(x)&=&(-1)^{n}\frac{2(2n+1)!}{n!}x{}_1F_1(-n;\fractje{3}{2};x^2)\
;\nn\\
L_n(x)&=&{}_1F_1(-n;1;x)\ ;\nn\\
L_n^k(x)&=&\frac{\Gamma(n+k+1)}{n!\Gamma(k+1)}{}_1F_1(-n;k+1;x)\
;\nn\\ T_n(x)&=&{}_2F_1\left(-n,n;\halfje;{1-x\over 2}\right)\ .
\ee\pagebreak[4]

\begin{thebibliography}{99}

\bibitem{MM} H. Margenau and G.M. Murphy, \emph{The Mathematics
of Physics and Chemistry}, D. van Nostrand Comp. Inc., 1943, 1956.
\bibitem{GR} I.S. Gradshteyn and I.W. Ryzhik,\emph{ Tables of Integrals,
Series and Products}, Acad. Press, New York, San Francisco,
London, 1965.
\bibitem{WB} W.W. Bell, \emph{Special Functions for Scientists and
Engineers}, D. van Nostrand Comp. Ltd., 1968.

\end{thebibliography}

\end{document}
