%==============================================================================
%== template for LATEX poster =================================================
%==============================================================================
%
%--A0 beamer slide-------------------------------------------------------------
\documentclass[final]{beamer} % use beamer
\usepackage[orientation=landscape,
            size=a0,          % poster size
            scale=1.6         % font scale factor
           ]{beamerposter}    % beamer in poster size
%
%--some needed packages--------------------------------------------------------
\usepackage[american]{babel}  % language 
\usepackage[utf8]{inputenc}   % std linux encoding
\usepackage[matrix,frame,arrow]{xypic}
\usepackage{color}
\usepackage{fancybox}
\usepackage{tikz}
%\newcommand{\ket}[1]{| #1 \rangle}
%\newcommand{\bra}[1]{\langle #1 |}
\newcommand{\braket}[1]{\langle #1 \rangle}
\newcommand{\ea}{\emph{et al.}}
\usetikzlibrary{shadows}

%\usepackage[usenames,dvipsnames]{xcolor}
\definecolor{brickred}{HTML}{B6321C}
\definecolor{olivegreen}{HTML}{3C8031}
\definecolor{navyblue}{HTML}{006EB8}
\definecolor{brownish}{RGB}{109,0,32}
\definecolor{redish}{RGB}{229,40,35}


\newcommand{\eq}[1]{\begin{align}#1\end{align}}


%\input{Qcircuit}


%
%==The poster style============================================================
\usetheme{cpbgposter}            % our poster style
%--set colors for blocks (without frame)---------------------------------------
  \setbeamercolor{block title}{fg=ngreen,bg=white}
  \setbeamercolor{block body}{fg=black,bg=white}
%--set colors for alerted blocks (with frame)----------------------------------
%--textcolor = fg, backgroundcolor = bg, dblue is the jacobs blue
  \setbeamercolor{block alerted title}{fg=white,bg=dblue!70}%frame color
  \setbeamercolor{block alerted body}{fg=black,bg=dblue!10}%body color
%
\newcommand{\dg}{^{\dagger}}
\newcommand{\hc}{\mathrm{H.c.}}
\newcommand{\cc}{\mathrm{c.c.}}
\newcommand{\ee}[1] {\mathrm{e}^{#1}}
\newcommand{\bra}[1] {\langle #1|}
\newcommand{\ket}[1] {|#1\rangle}
\newcommand{\name}{Self- and cross- phase modulation in the generation of bright twin beams using SPDC
}

%==Titel, date and authors of the poster=======================================
%\title{Self-calibrating tomography for multi-dimensional systems}
\title{\name}
\author{Nicol\'as Quesada$^1$ and J.E Sipe $^2$}
\institute[]
          {$^1$Department of Physics and Astronomy, Macquarie University \\  $^2$Department of Physics,  University of Toronto}
%\date{November 13, 2012}
%
%==some usefull qm commands====================================================
%  |x>
%\newcommand{\ket}[1]{\left\vert#1\right\rangle}
\newcommand{\dbra}[1]{\langle \langle #1|}
\newcommand{\dket}[1]{|#1 \rangle \rangle}
%  <x|
%\newcommand{\bra}[1]{\left\langle#1\right\vert}
%  <x|y>
%\newcommand{\braket}[2]{\left< #1 \vphantom{#2}\,
%                        \right\vert\left.\!\vphantom{#1} #2 \right>}
%  <x|a|y>
\newcommand{\ks}[1]{\bar k_{#1}+\delta k_{#1}}
\newcommand{\ws}[1]{\bar \omega_{#1}+\delta \omega_{#1}}

\newcommand{\sandwich}[3]{\left< #1 \vphantom{#2 #3} \right|
                          #2 \left|\vphantom{#1 #2} #3 \right>}
%  d/dt
\newcommand{\ddt}{\frac{d}{dt}}
%  D/Dx
\newcommand{\pdd}[1]{\frac{\partial}{\partial#1}}
%  |x|
\newcommand{\abs}[1]{\left\vert#1\right\vert}
%  k_{x}
\newcommand{\kv}[1]{\mathbf{k}_{#1}}
%==============================================================================
%==the poster content==========================================================
%==============================================================================
\usepackage{exscale}
\begin{document}
%--the poster is one beamer frame, so we have to start with:
\begin{frame}[t]
%--to seperate the poster in columns we can use the columns environment
 \begin{columns}[t] % the [t] options aligns the columns content at the top
%--the left column-------------------------------------------------------------
  \begin{column}{0.33\paperwidth}% the right size for a 3-column layout
%--abstract block--------------------------------------------------------------
   \begin{alertblock}{Abstract}
\begin{minipage}{0.18\textwidth}
\begin{figure}[H]
\includegraphics[width=1 \linewidth]{phase_qr.pdf}
\end{figure}
\end{minipage} \hfill
\begin{minipage}{0.8\textwidth}
We introduce a simple methodology to calculate the effects of self- and cross-phase modulation in SPDC photon generation. We show that these processes make SPDC less efficient in the low spatio-temporal mode number limit.
\end{minipage}
   \end{alertblock}
   \vskip2ex
\setbeamercolor{block alerted title}{fg=white,bg=dblue!70}
\setbeamercolor{block alerted body}{fg=black,bg=white!30}


\begin{alertblock}{Phase matching  and energy conservation in photon generation}
In recent experiments \cite{harder,lemieux} bright sources of squeezed light have been engineered using $\chi^{(2)}$ nonlinearities. These sources cannot be described perturbatively and one needs to worry about higher order effects (e.g. $\chi^{3}$) that could also affect the states generated.
%\hline
\begin{block}{$\chi^2$ nonlinearities}
Three modes $A,B,C$ interact via a $\chi^{(2)}$ nonlinear susceptibility
\begin{align*}
H_2\sim\int dz \chi^{(2)}(z) \left( \sum_{M=A,B,C} a_M(\omega_M)e^{i (\ks{M})z-i (\ws{M}) t} + \hc \right)^{3}
\end{align*}
where $\bar k_M = \bar \omega_M/\underbrace{v_M^{(p)}}_{\text{phase vel.}}$ and $\delta k_M = \delta \omega_M/\underbrace{v_M^{(g)}}_{\text{group vel.}}$.
For
\begin{align*}
\bar \omega_A + \bar \omega_B = \bar \omega_C \text{ and }  \bar k_A + \bar k_B = \bar k_C ,
\end{align*} 
the only term that is relevant is $H_{\text{SPDC}}\sim a_A ^\dagger a_B^\dagger a_C+\hc$
\end{block}
%\hline
\begin{block}{$\chi^3$ nonlinearities}
Three modes $A,B,C$ interact via a $\chi^{(3)}$ nonlinear susceptibility
\begin{align*}
H_3\sim \int dz \chi^{(3)}(z) \left( \sum_{M=A,B,C} a_M(\omega_M)e^{i (\ks{M})z-i (\ws{M}) t} + \hc \right)^{4}
\end{align*}
\begin{itemize} 
\item \textcolor{brownish}{Terms like $a_A^\dagger a_A a_C^\dagger a_C $ always survive.}
\item If mode $C$ is a strong coherent state it will create a nonlinear index of refraction for itself $H_{\text{SPM}} \sim a_C^\dagger a_C a^\dagger_C a_C$: \emph{self-phase modulation (SPM)}.
\item It will also cause a nonlinear index of refraction for the other modes $H_{\text{XPM}} \sim a_A^\dagger a_A a^\dagger_C a_C$: \emph{cross-phase modulation (XPM)}
\end{itemize}
\end{block}
\end{alertblock}

\end{column}
%===big rightcolumn=============================================================


\begin{column}{0.34\paperwidth} 
\setbeamercolor{block alerted title}{fg=white,bg=dblue!70}
\setbeamercolor{block alerted body}{fg=black,bg=white!30}
\begin{alertblock}{Accounting for the effects of SPM and XPM}
In the strong driven regime one needs to consider both the photon generation via SPDC and the SPM and XPM:
\begin{align*}
H=H_{\text{SPDC}}+H_{\text{XPM}}+H_{\text{SPM}}.
\end{align*}
%One can go to an interaction picture with respect to $H_{\text{XPM}}+H_{\text{SPM}}$ 
If the pump is a strong coherent state $a_C(\omega_C)\to \langle a_C(\omega_C) \rangle = \alpha_C(\omega_C)$
\begin{align*}
\label{hi}
 H_{\text{SPDC}} \sim  \int dz \chi^{(2)}(z) \int d\omega _A d\omega
_B d\omega _P e^{i(\delta k _P-\delta k_A-\delta k_B) z}  \\
e^{i (\delta \omega_A+\delta \omega_B-\delta \omega_P) t} \alpha_C(\omega_C)  a_A^{\dagger }(\omega_A)  a_B^{\dagger }(\omega _B),
\end{align*}
\begin{itemize}
\item Free propagation is accounted for by the phases $e^{i (\delta k z -\delta \omega t)}$.

\item \textcolor{brownish}{Go to an interaction picture with respect to $H_{\text{XPM}}$ and $H_{\text{SPM}}$.}
\item \textcolor{brownish}{This is done by including an extra propagation phase }
\end{itemize}
\begin{align*}
\Gamma_{A/B}(z,t)=2 \int^t dt''\chi^{(3)}(z-v_{A/B} (t-t'')) |\phi_C(z-v_{A/B}(t-t''),t'')|^2,
\end{align*}
\begin{itemize}
\item  $\chi^3(z)$ describes the spatial profile of the XPM nonlinearity. 
\item For the pump mode undergoing SPM, one should replace $v_{A/B}$ by $v_P$ and remove the prefactor of 2.
\item $\phi_P(z,t)=\int \frac{d\omega_P}{\sqrt{2 \pi v_P}} \alpha(\omega_P) e^{-i (\omega_P- \bar \omega_P)(t-z/v_P)}$ is the pump profile in real space.
\end{itemize}
Upon the inclusion of SPM and XPM and taking 
\begin{align*}
\alpha(\omega_c)=\frac{\tau}{\sqrt{\pi}} \exp(-\tau^2 \delta \omega_c^2)
\end{align*} 
we can rewrite the interaction picture Hamiltonian as
\begin{align*}
H_I(t)=-\hbar \varepsilon\int d\omega_A d\omega_B d\omega_P F(\omega_A,\omega_B,\omega_P) e^{i(\delta \omega_A +\delta \omega_B - \delta \omega_P)t} \\
a^\dagger(\omega_A) b^\dagger(\omega_B)+\hc
\end{align*}
where $F(\omega_A,\omega_B,\omega_P)=\mathcal{F}\left(-(\delta \omega_A/v_A+\delta \omega_B/v_B),\omega_P\right),$  and
\begin{align*}
\mathcal{F}(q,\omega_P)&=\int \frac{dz}{\sqrt{2 \pi}} \frac{dt}{\sqrt{2 \pi}} \mathcal{G}(z,t) e^{i (q z + \delta \omega_P t)}, \\ \mathcal{G}(z,t)&= \frac{\chi^{(3)}(z)}{L}  e^{-(z-v_P t)^2/(4 v_P^2 \tau^2)} e^{-i \Gamma(z,t)}
\end{align*}
are a 2D Fourier transform pair in the spaces $(q,\omega_P)$,  $(z,t)$ and 
\begin{align*}
\Gamma=\Gamma_A+\Gamma_B-\Gamma_P.
\end{align*}
The joint spectral amplitude is: 
\begin{align*}
 J_1(\omega_A,\omega_B)=F(\omega_A,\omega_B,\omega_A+\omega_P).
\end{align*}

\end{alertblock}

\end{column}


  \begin{column}{0.30\paperwidth} 
\setbeamercolor{block alerted title}{fg=white,bg=dblue!70}
\setbeamercolor{block alerted body}{fg=black,bg=white!30}
\begin{alertblock}{Consequences for SPDC}
\begin{itemize}
\item The spectral structure: the Schmidt decomposition \\
 $J_1(\omega_A,\omega_B)=\sum_\lambda s_\lambda f_\lambda (\omega_A) j_\lambda(\omega_B)$. 
\item The Schmidt coefficients satisfy $\sum_\lambda s_\lambda^2=\int d \omega_A d\omega_B |J_1(\omega_A,\omega_B)|^2$.
\item XPM and SPM will not affect the norm of the JSA since
\begin{align*}
\int d \omega_A d\omega_B |J_1(\omega_A,\omega_B)|^2&= \mathcal{J} \int d \omega_P dq |\mathcal{F}(q,\omega_P)|^2 \\
&= \mathcal{J}  \int dz dt|\mathcal{G}(q,\omega_P)|^2,
\end{align*}
where $\mathcal{J}=\frac{v_A v_B}{|v_A-v_B|}$ is the Jacobian of  $(\omega_P=\omega_A+\omega_B,q=-(\delta \omega_A/v_A+\delta \omega_B/v_B))$ and we used Parseval's theorem.
\item The number of photons will be smaller than than would be predicted if XPM and SPM where ignored since
\begin{align*}
\braket{ \tilde N^{\text{no XPM-SPM}}_{A/B}}=\sum_{\lambda} \sinh^2\left( \sqrt{s_\lambda^2} \right) \leq  \sinh^2 \left( \sqrt{ \sum_{\lambda} s_\lambda^2}\right) \\
=\sinh^2 \left( \sqrt{ \tilde s_0^2}\right)=\braket{N_{A/B}}.
\end{align*}
In the last derivation time-ordering effects have been assumed to be negligible \cite{nico1}.\\
\textcolor{brownish}{XPM and SPM leads to additional Schmidt modes and reduces estimulation}
\end{itemize}

\end{alertblock}

\begin{alertblock}{References}
\begin{thebibliography}{99}
\bibitem{harder} G. Harder \ea, ``Single-Mode Parametric-Down-Conversion States with 50 Photons as a Source for Mesoscopic Quantum Optics'', Phys. Rev. Lett. {\bf 116}, 143601 (2016).
\bibitem{lemieux} S. Lemieux \ea, ``Engineering the Frequency Spectrum of Bright Squeezed Vacuum via Group Velocity Dispersion in an SU(1,1) Interferometer'', Phys. Rev. Lett. {\bf 117}, 183601 (2016).
\bibitem{nico1} N. Quesada and J.E. Sipe, ``Time-ordering effects in the generation of entangled photons using nonlinear optical processes'', Phys. Rev. Lett. {\bf 114} , 093903 (2015).
\end{thebibliography}
\end{alertblock}

  \end{column}
 \end{columns}
\end{frame}

\end{document}
