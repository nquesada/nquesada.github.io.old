%==============================================================================
%== template for LATEX poster =================================================
%==============================================================================
%
%--A0 beamer slide-------------------------------------------------------------
\documentclass[final]{beamer} % use beamer
\usepackage[orientation=landscape,
            size=a0,          % poster size
            scale=1.6         % font scale factor
           ]{beamerposter}    % beamer in poster size
%
%--some needed packages--------------------------------------------------------
\usepackage[american]{babel}  % language 
\usepackage[utf8]{inputenc}   % std linux encoding
\usepackage[matrix,frame,arrow]{xypic}
\usepackage{color}
\usepackage{fancybox}
\usepackage{tikz}
\usepackage{float}

\usetikzlibrary{shadows}

%\usepackage[usenames,dvipsnames]{xcolor}
\definecolor{brickred}{HTML}{B6321C}
\definecolor{olivegreen}{HTML}{3C8031}
\definecolor{navyblue}{HTML}{006EB8}


\definecolor{ShadowColor}{RGB}{109,0,32}


\definecolor{brownish}{RGB}{109,0,32}
\definecolor{redish}{RGB}{229,40,35}

%\input{Qcircuit}


%
%==The poster style============================================================
\usetheme{cpbgposter}            % our poster style
%--set colors for blocks (without frame)---------------------------------------
  \setbeamercolor{block title}{fg=ngreen,bg=white}
  \setbeamercolor{block body}{fg=black,bg=white}
%--set colors for alerted blocks (with frame)----------------------------------
%--textcolor = fg, backgroundcolor = bg, dblue is the jacobs blue
  \setbeamercolor{block alerted title}{fg=white,bg=dblue!70}%frame color
  \setbeamercolor{block alerted body}{fg=black,bg=dblue!10}%body color
%
\newcommand{\dg}{^{\dagger}}
\newcommand{\hc}{\mathrm{h.c.}}
\newcommand{\cc}{\mathrm{c.c.}}
\newcommand{\ee}[1] {\mathrm{e}^{#1}}
\newcommand{\bra}[1] {\langle #1|}
\newcommand{\ket}[1] {|#1\rangle}
\newcommand{\name}{Why you should \emph{not} use the electric field \\
to quantize in nonlinear optics
}

%==Titel, date and authors of the poster=======================================
\title{\name}
\author{Nicol\'as Quesada$^1$ and J.E Sipe $^2$}
\institute[]
          {$^1$Department of Physics and Astronomy, Macquarie University \\  $^2$Department of Physics,  University of Toronto}
%\date{November 13, 2012}
%
%==some usefull qm commands====================================================
%  |x>
%\newcommand{\ket}[1]{\left\vert#1\right\rangle}
\newcommand{\dbra}[1]{\langle \langle #1|}
\newcommand{\dket}[1]{|#1 \rangle \rangle}
%  <x|
%\newcommand{\bra}[1]{\left\langle#1\right\vert}
%  <x|y>
%\newcommand{\braket}[2]{\left< #1 \vphantom{#2}\,
%                        \right\vert\left.\!\vphantom{#1} #2 \right>}
%  <x|a|y>
\newcommand{\sandwich}[3]{\left< #1 \vphantom{#2 #3} \right|
                          #2 \left|\vphantom{#1 #2} #3 \right>}
%  d/dt
\newcommand{\ddt}{\frac{d}{dt}}
%  D/Dx
\newcommand{\pdd}[1]{\frac{\partial}{\partial#1}}
%  |x|
\newcommand{\abs}[1]{\left\vert#1\right\vert}
%  k_{x}
\newcommand{\kv}[1]{\mathbf{k}_{#1}}
%==============================================================================
%==the poster content==========================================================
%==============================================================================
\usepackage{exscale}
%\usepackage{wrapfigure}
%\usepackage{minipage}
\begin{document}
%--the poster is one beamer frame, so we have to start with:
\begin{frame}[t]
%--to seperate the poster in columns we can use the columns environment
 \begin{columns}[t] % the [t] options aligns the columns content at the top
%--the left column-------------------------------------------------------------
  \begin{column}{0.32\paperwidth}% the right size for a 3-column layout
%--abstract block--------------------------------------------------------------
   \begin{alertblock}{Abstract}
\begin{minipage}{0.18\textwidth}
\begin{figure}[H]
\includegraphics[width=1 \linewidth]{quantization_qr.pdf}
\end{figure}
\end{minipage} \hfill
\begin{minipage}{0.8\textwidth}
We show that using the electric field as a canonical quantization variable in nonlinear optics leads to incorrect expressions for the squeezing parameters in SPDC and conversion rates in frequency conversion.

\end{minipage}
   \end{alertblock}
   \vskip2ex

\setbeamercolor{block alerted title}{fg=white,bg=dblue!70}
\setbeamercolor{block alerted body}{fg=black,bg=white!30}


\begin{alertblock}{Quantization recipe}
%The quantization of the EM field as a long history going back to Born and Infeld[1]. 

\begin{enumerate}
\item Write a Hamiltonian $H$ in terms of the relevant fields $\mathbf{F}$ so that it is numerically equal to the energy.
\item Define a set of canonical commutation relations for the fields.
\item \textcolor{brownish}{Check that the Heisenberg equations of motion (EOM) }
\begin{align}
\dot{ \mathbf{F} }   = \frac{[\mathbf{F},H]}{i \hbar}
\end{align}
\textcolor{brownish}{are the Equations of Motion for the fields} \cite{sipe}.\\
\end{enumerate}


\bigskip
For the Electromagnetic field (EMF) for example \cite{sipe,drummond}:
\item The Hamiltonian is 
\begin{align}
H&=\int d^3 \mathbf{r} \left(\frac{\mathbf{B}^2}{2 \mu_0}+  \epsilon_0 \mathbf{E} \cdot \left\{ \frac{1+\chi^{(1)}}{2}\mathbf{E} + \sum_{n \geq 2} \frac{n}{n+1}  \mathbf{\chi}^{(n)} : \mathbf{E}^n \right\} \right)\\
&=\int d^3\mathbf{r} \left(\frac{\mathbf{B}^2}{2 \mu_0} +\sum_{n\geq 1}\frac{1}{n+1}\mathbf{D}\cdot \mathbf{\eta}^{(n)} : \mathbf{D}^n\right)
\end{align}
The (only nonzero) commutation relation is 
\begin{align}
[D^i(\mathbf{r}),B^j(\mathbf{r'})]=i\hbar \epsilon^{ijk} \frac{\partial}{\partial r^k} \delta(\mathbf{r}-\mathbf{r}')
\end{align}
and the Heisenberg EOMs are Maxwell's equations
\begin{align}
\dot{\mathbf{D}} = \nabla \times  \mathbf{B}/\mu_0, \quad & \dot{\mathbf{B}} = -\nabla \times  \mathbf{E}
\end{align}
if the following constitutive relation is assumed 
\begin{align}\label{const}
\mathbf{E}=\frac{\mathbf{D}-\mathbf{P}}{\epsilon_0} = \sum_{n}\mathbf{\eta}^{(n)}: \mathbf{D}^n
\end{align}
which defines macroscopic polarization in terms of $\mathbf{D}$ and $\eta^{(n)}$. \\
Also can define in terms of $\mathbf{E}$ and nonlinear susceptibilities $\chi^{(n)}$
\begin{align}
\eta^{(1)} = \epsilon_0^{-1} (1+\chi^{(1)})^{-1}, \\
\eta^{(2)}=-\epsilon_0 \eta^{(1)} \chi^{(2)}: \eta^{(1)} \eta^{(1)}
\end{align}
{\color{white} empty}
\end{alertblock}

\end{column}
%===big rightcolumn=============================================================


\begin{column}{0.34\paperwidth} 
\setbeamercolor{block alerted title}{fg=white,bg=dblue!70}
\setbeamercolor{block alerted body}{fg=black,bg=white!30}
\begin{alertblock}{What goes wrong with the electric field?}
%\colorbox{redish}{
{\Large
\begin{eqnarray*}
\shadowbox{
\parbox[c]{18.5em}{
\quad \ \textcolor{redish}{Can you quantize with  $\mathbf{E}$ and still satisfy \\ Maxwell's Equations in a nonlinear medium?}}}
\end{eqnarray*}
}

%}
%\begin{center}
%  \begin{tikzpicture}
%    \node [copy shadow={fill=black,shadow xshift=-0.5ex,shadow yshift=-0.5ex},
%            fill=white!20,draw=black,thick] {Can one quantize with $\mathbf{E}$ and still get \\
%Maxwell's Equations in a nonlinear medium?};
%  \end{tikzpicture}
%\end{center}

%\begin{block}{Can one quantize with $\mathbf{E}$ and still get Maxwell's Equations in a nonlinear medium?}
Let's try to write
\begin{align}
\mathbf{E}(\mathbf{r})& = \sum_J \int d{\mathbf{k}} \ \mathbf{e}_J(\mathbf{k},\mathbf{r}) a_{J}(\mathbf{k})+\text{H.c.} = \text{poly}_1(  a_J(\mathbf{k}),  a_J(\mathbf{k})^\dagger)\nonumber\\
\mathbf{B}(\mathbf{r})& = \sum_J \int d{\mathbf{k}} \ \mathbf{b}(\mathbf{k},\mathbf{r}) a_J(\mathbf{k})+\text{H.c.}=\text{poly}_1(  a_J(\mathbf{k}),  a_J(\mathbf{k})^\dagger), \nonumber\\
&[ a_J(\mathbf{q}),  a_{J'}^\dagger(\mathbf{q'})]=\delta_{J J'}\delta(\mathbf{q'}- \mathbf{q}) 
\end{align}
%\begin{center}
%\begin{tabular}{c |c |c }
% Label & 1D & 3D \\ 
%\hline
% $J$ & Spatial Mode& Polarization \\  
%\hline
% $\mathbf{k}$ & $k_z$ & $(k_x,k_y,k_z) $
%\end{tabular}
%\end{center}
$\text{poly}_n(x,y)$: polynomial of degree $n$ in variables $x,y$. Then,
\begin{align}
H = \text{poly}_{n+1}(  a_J(\mathbf{k}),  a_J(\mathbf{k})^\dagger) (\text{if } \chi^{(n)} \neq 0)
\end{align}
Now let us look at Faraday's law of induction
\begin{align}\label{cont}
-\nabla \times \mathbf{E} &= [\mathbf{B},H] /(i \hbar)= \dot{ \mathbf{B}} \nonumber\\
\text{poly}_1(  a_J(\mathbf{k}),  a_J(\mathbf{k})^\dagger) &\stackrel{?}{=} [ \text{poly}_1(  a_J(\mathbf{k}),  a_J(\mathbf{k})^\dagger), \text{poly}_{n+1}(  a_J(\mathbf{k}),  a_J(\mathbf{k})^\dagger)]\nonumber\\
\text{poly}_1(  a_J(\mathbf{k}),  a_J(\mathbf{k})^\dagger) &\stackrel{?}{=} \text{poly}_n(  a_J(\mathbf{k}),  a_J(\mathbf{k})^\dagger) 
\end{align}
The electric field has to be a \emph{nonlinear} function of $ a_J(\mathbf{k}),  a_J(\mathbf{k})^\dagger $.\\
{\Large
\begin{equation*}
\shadowbox{
\textcolor{redish}{\text{No, you cannot.}}
}
\end{equation*}
}
\end{alertblock}
\begin{alertblock}{Quantization with $\mathbf{D}$}
If quantization is done with $\mathbf{D}$ and $\mathbf{B}$ then
\begin{align}
\mathbf{D}(\mathbf{r})& = \sum_J \int d{\mathbf{k}} \ \mathbf{d}_J(\mathbf{k},\mathbf{r}) a_{J}(\mathbf{k})+\text{H.c.} 
\end{align}
Constitutive relation Eq. (\ref{const}) gives $\mathbf{E}$ as a \emph{nonlinear} function of the creation and annihilation operators
\begin{align}\label{Epow}
\mathbf{E}&=\frac{\mathbf{D}-\mathbf{P}}{\epsilon_0} = \underbrace{\eta^{(1)} \mathbf{D}}_{\mathbf{E}_{\text{linear}}}+\sum_{n>1}\mathbf{\eta}^{(n)}: \mathbf{D}^n\\
&=\sum_{n}\mathbf{\eta}^{(n)}: \left(\sum_J \int d{\mathbf{k}} \ \mathbf{d}_J(\mathbf{k},\mathbf{r}) a_{J}(\mathbf{k})+\text{H.c.}\right)^n
\end{align}
This same conclusion is found by looking at Eq. (\ref{cont}).
\end{alertblock}
\end{column}


  \begin{column}{0.30\paperwidth} 
\setbeamercolor{block alerted title}{fg=white,bg=dblue!70}
\setbeamercolor{block alerted body}{fg=black,bg=white!30}
\begin{alertblock}{Consequences for bright SPDC}
%Spontaneous Parametric Down-Conversion in 1-D geometry:
\begin{itemize}
\item 3 guided modes on a waveguide labeled $A,B,C$ 
\item Phase and energy matched
\begin{align}
\bar k_C= \bar k_A+\bar k_B \text{ and }  \bar \omega_C = \bar \omega_A + \bar \omega_B.
\end{align}
\item Nonlinear part of the Hamiltonian in the interaction picture with respect to free propagation is
\begin{align}
\label{eq}
H_I(t)=&\hbar \theta' \int dk_A dk_B dk_C  e^{i (\omega_A(k_A)+\omega_B(k_B)-\omega_C(k_C)) t} 
\\& \times \Phi(k_A,k_b,k_c) a_A^\dagger(k_A) a_B^\dagger(k_B) a_C (k_C) +\text{H.c.}\nonumber
\end{align}
\end{itemize}
{\Large
\begin{eqnarray*}
\shadowbox{
\parbox[c]{17.5em}{
\textcolor{redish}{Using $\mathbf{E}_\text{linear}$ the  interaction picture Hamiltonian is a factor of two larger than what is predicted with the correct quantization.}}}
\end{eqnarray*}
}
\textcolor{brownish}{Compare factor of $\frac{n}{n+1}$ in Eq. (2) vs. $\frac{1}{n+1}$ in Eq. (3).}
\end{alertblock}

\begin{alertblock}{Effect on twin beams generation}
%The correct expression for this Hamiltonian is on the other a factor of to smaller. 
\begin{itemize}
\item Mode $C$ in coherent state with $ N \gg 1$ photons: 
\begin{align}
a_C(k_C) \sim \langle a_C(k_C) \rangle=\sqrt{N} \alpha(k_C)
\end{align}
\item The mean number of photons in the twin beams is 
\begin{align}
\langle N_{A/B} \rangle=\sinh^2(\theta \sqrt{N}).
\end{align} 
\item With $\mathbf{E}_{\text{linear}}$: $\langle N_{A/B \text{ linear}} \rangle=4\langle N_{A/B} \rangle^2.$
\end{itemize}
%\end{block}
\end{alertblock}

\begin{alertblock}{References}
\begin{thebibliography}{99}
\bibitem{sipe} J.E. Sipe, \emph{et al.}, ``Effective field theory for the nonlinear optical properties of photonic crystals'', Phys. Rev. E {\bf 69}, 016604 (2004).
\bibitem{drummond} P.D. Drummond and M. Hillery, \emph{The Quantum Theory of Nonlinear Optics}, (Cambridge, 2014).
\bibitem{hillery} M. Hillery, \emph{et al.}, ``Quantization of electrodynamics in nonlinear media'', Phys. Rev. A {\bf 30}, 1860 (1984).
\end{thebibliography}
\end{alertblock}

  \end{column}
 \end{columns}
\end{frame}

\end{document}
