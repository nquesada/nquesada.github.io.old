\documentclass[letterpaper,12pt]{article}
\usepackage[spanish]{babel}


\textwidth = 16.5 cm
\textheight = 21.9 cm
\oddsidemargin = 0.0 cm
\evensidemargin = 0.0 cm
\topmargin = 1.0 cm
\headheight = 0.0 cm
\headsep = 0.0 cm 

\title{Transformaci\'on Can\'onica para el Oscilador Arm\'onico}
\author{Nicol\'as Quesada M.\\{\small \sf Instituto de F\'isica, Universidad de Antioquia}}
\date{}
\begin{document}

\maketitle

Se tiene la transformaci\'on :
\begin{eqnarray}
a&=&\frac{e^{i  \omega t} (m \omega q + i p )}{
   \sqrt{ 2 m \omega }} 
\label{eqn1}
\\
a^*&=&\frac{e^{-i  \omega t} (m \omega q - i p)}{
   \sqrt{ 2m \omega }} 
\label{eqn2}
\end{eqnarray}
Se pide mostrar que la transformaci\'on $(q,p)\longmapsto(a,ia^{*})$ es can\'onica y exhibir su funci\'on generatriz.\\

Las ecuaciones (\ref{eqn1}) y (\ref{eqn2}) pueden ser invertidas para obtener:
\begin{eqnarray}
q&=&\frac{ a e^{-i \omega t }+a^* e^{i \omega t}}{\sqrt{2 m \omega}}\\
p&=&i  \sqrt{\frac{m \omega}{2}} \left(a^* e^{i \omega t}-a e^{-i \omega t}\right)
\end{eqnarray}

Se buscar\'a una funci\'on generatriz de primer tipo. Con ella se pueden escribir las ecuaciones auxiliares:
\begin{eqnarray}
p&=&\frac{\partial F_1}{\partial q} \\
i a^*&=&-\frac{\partial F_1}{\partial a} 
\end{eqnarray}
De la ecuaci\'on que define a $a$ se puede obtener a $p$ en terminos de $a$ y $q$ asi:
\begin{eqnarray}
p=i \left(m q \omega - e^{-i \omega t} \sqrt{2 m \omega} a \right)
\end{eqnarray}
y sustituirlo en la primera ecuaci\'on auxiliar e integrar para obtener:
\begin{eqnarray}
F_1=i \left(\frac{1}{2} m q^2 \omega - a e^{-i 
   \omega t} q \sqrt{2 m \omega }\right)+h(a)
\end{eqnarray}
La anterior ecuaci\'on se puede derivar parcialmente con respecto a $a$ y compararla con la segunda ecuaci\'on auxiliar (De la ecuaci\'on que define a $q$ en terminos de $a$ y $a^*$ podemos despejar esta \'ultima):
\begin{equation}
\frac{\partial F_1}{\partial a}=h'(a)-i  e^{-i  \omega t} q \sqrt{2 m \omega }=-i a^*=i \left(a e^{-2 i  \omega t}- e^{-i  \omega t } q \sqrt{2 m \omega }\right)
\end{equation}


Para obtener:
\begin{equation}
h'(a)= i a e^{-2 i \omega  t} \longrightarrow h(a)=\frac{1}{2} i a^2 e^{-2 i \omega t }
\end{equation}
Finalmente la funci\'on buscada es:

\begin{eqnarray}
F_1=\frac{1}{2} i \left(e^{-2 i t \omega } a^2-2 e^{-i t \omega } a \sqrt{2 m \omega } q +m q^2 \omega \right)
\end{eqnarray}
Con los anterior se ve que la transformaci\'on dada por (\ref{eqn1}) y (\ref{eqn2}) es can\'onica. Veamos como queda el hamiltoniano del oscilador arm\'onico en t\'erminos de las nuevas variables can\'onicas $(a,ia^*)$.\\
El hamiltoniano del oscilador arm\'onico en las variables $(q,p)$ es:
\begin{equation}
H=\frac{p^2}{2m}+\frac{m \omega^2 q^2}{2}
\end{equation}

Por otro lado el Hamiltoniano  transformado es:
\begin{equation}
H'(a,a^*)=H(a,a^*)+\frac{\partial F}{\partial t}
\end{equation}
Adem\'as tenemos que:
\begin{eqnarray}
a^* a=\frac{H}{\omega} \Longrightarrow H=\omega  a^* a \\
\frac{\partial F}{\partial t}=\omega \left( a^2 e^{-2 i \omega t}  - a e^{-i  \omega  t  } q   \sqrt{2 m \omega } \right)
\end{eqnarray}
que al cambiar a $q$ en t\'erminos de $a$ y $a^*$ se simplifica enormemente para dar:
\begin{equation}
\frac{\partial F}{\partial t}=- \omega a^* a 
\end{equation}
Finalmente $H'=0$. \\
\\
Estas transformaciones han hecho que las nuevas coordenadas y momentos sean constantes y que adem\'as estos tengan por valor la exponencial de una fase debida a las condiciones iniciales por una funci\'on de la energ\'ia. Mas exactamente:
\begin{eqnarray}
 q=A \cos (\omega t-\phi )\Longrightarrow   p=-A m \omega  \sin (\omega t-\phi)
\end{eqnarray}
 entonces 
\begin{eqnarray}
a&=&\frac{A e^{i \phi } \sqrt{m \omega }}{\sqrt{2}}=\frac{e^{i \phi } \sqrt{E}}{\sqrt{ \omega}}\\
a^*&=&\frac{A e^{-i \phi } \sqrt{m \omega }}{\sqrt{2}}=\frac{ e^{-i \phi } \sqrt{E}}{\sqrt{ \omega} }
\end{eqnarray}
\end{document}
