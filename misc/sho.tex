\documentclass[spanish,12pt]{shreyasnotes}

\title{Transformaci\'on Can\'onica para el Oscilador Arm\'onico}
\author{Nicol\'as Quesada M.\\{\small \sf Instituto de F\'isica, Universidad de Antioquia}}
\date{}
\begin{document}

\maketitle

Se tiene la transformaci\'on 
\begin{subequations}
\begin{align}
a&=\frac{ m \omega q + i p }{
   \sqrt{ 2 m \omega }} ,
\label{eqn1}
\\
a^*&=\frac{ m \omega q - i p}{
   \sqrt{ 2m \omega }} .
\label{eqn2}
\end{align}
\end{subequations}
Se pide mostrar que la transformaci\'on $(q,p)\longmapsto(a,ia^{*})$ es can\'onica y exhibir su funci\'on generatriz.\\

Las ecuaciones (\ref{eqn1}) y (\ref{eqn2}) pueden ser invertidas para obtener:
\begin{subequations}
\begin{align}
q&=\frac{ a +a^* }{\sqrt{2 m \omega}},\\
p&=i  \sqrt{\frac{m \omega}{2}} \left(a^* -a \right).
\end{align}
\end{subequations}

Se buscar\'a una funci\'on generatriz de primer tipo. Con ella se pueden escribir las ecuaciones auxiliares
\begin{subequations}
\begin{align}
p&=\frac{\partial F_1}{\partial q}, \\
i a^*&=-\frac{\partial F_1}{\partial a} .
\end{align}
\end{subequations}
De la ecuaci\'on que define a $a$ se puede obtener a $p$ en terminos de $a$ y $q$ asi
\begin{align}
p=i \left(m q \omega - \sqrt{2 m \omega} a \right)
\end{align}
y sustituirlo en la primera ecuaci\'on auxiliar e integrar para obtener
\begin{align}
F_1=i \left(\frac{1}{2} m q^2 \omega - a  q \sqrt{2 m \omega }\right)+h(a).
\end{align}
La anterior ecuaci\'on se puede derivar parcialmente con respecto a $a$ y compararla con la segunda ecuaci\'on auxiliar (de la ecuaci\'on que define a $q$ en terminos de $a$ y $a^*$ podemos despejar esta \'ultima)
\begin{align}
\frac{\partial F_1}{\partial a}=h'(a)-i  q \sqrt{2 m \omega }=-i a^*=i \left(a - q \sqrt{2 m \omega }\right),
\end{align}
para obtener
\begin{align}
h'(a)= i a  \longrightarrow h(a)=\frac{1}{2} i a^2 .
\end{align}
Finalmente la funci\'on buscada es
\begin{align}
F_1=\frac{1}{2} i \left( a^2-2  a \sqrt{2 m \omega } q +m q^2 \omega \right).
\end{align}
Con lo anterior se ve que la transformaci\'on dada por (\ref{eqn1}) y (\ref{eqn2}) es can\'onica. Veamos como queda el hamiltoniano del oscilador arm\'onico en t\'erminos de las nuevas variables can\'onicas $(a,ia^*)$.\\
El hamiltoniano del oscilador arm\'onico en las variables $(q,p)$ es
\begin{align}
H=\frac{p^2}{2m}+\frac{m \omega^2 q^2}{2}.
\end{align}

Por otro lado el Hamiltoniano  transformado es
\begin{align}
H'(a,a^*)=H(a,a^*)+\frac{\partial F}{\partial t}.
\end{align}
Adem\'as tenemos que
\begin{align}
a^* a=\frac{H}{\omega} \Longrightarrow H=\omega  a^* a \\
\frac{\partial F}{\partial t}=0.
\end{align}
Finalmente 
\begin{align}
H'=\omega  a^* a.
\end{align}
Como $(a, ia^*)$ son variables can\'onicas se cumple que $\{a,i a^*\}=1$ y que su din\'amica est\'a dada por las ecuaciones de Hamilton. Con lo anterior es directo calcular las ecuaciones de movimento del sistema:
\begin{subequations}
\begin{align}
 \frac{d}{dt} a&=\{a, \omega a^*a\} =\omega \left(\{a,a^*\}a+a^*\{a,a\} \right)=-i \omega a,\\
 \frac{d}{dt} a^*&=\{a^*, \omega a^*a\}=\omega \left(\{a^*,a^*\}a+a^*\{a^*,a\} \right)=i \omega a^*,
\end{align}
\end{subequations}
y su soluci\'on
\begin{subequations}
\begin{align}
 a(t)&=a(0) e^{-i \omega t}\\
a^*(t)&=a^*(0) e^{i \omega t}
\end{align}
\end{subequations}
\end{document}
